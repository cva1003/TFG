\documentclass[a4paper,12pt,twoside]{memoir}

% Castellano
\usepackage[spanish,es-tabla]{babel}
\selectlanguage{spanish}
\usepackage[utf8]{inputenc}
\usepackage[T1]{fontenc}
\usepackage{lmodern} % Scalable font
\usepackage{microtype}
\usepackage{placeins}

\RequirePackage{booktabs}
\RequirePackage[table]{xcolor}
\RequirePackage{xtab}
\RequirePackage{multirow}

% Links
\PassOptionsToPackage{hyphens}{url}\usepackage[colorlinks]{hyperref}
\hypersetup{
	allcolors = {red}
}

% Ecuaciones
\usepackage{amsmath}

% Rutas de fichero / paquete
\newcommand{\ruta}[1]{{\sffamily #1}}

% Párrafos
\nonzeroparskip

% Huérfanas y viudas
\widowpenalty100000
\clubpenalty100000

\let\tmp\oddsidemargin
\let\oddsidemargin\evensidemargin
\let\evensidemargin\tmp
\reversemarginpar

% Imágenes

% Comando para insertar una imagen en un lugar concreto.
% Los parámetros son:
% 1 --> Ruta absoluta/relativa de la figura
% 2 --> Texto a pie de figura
% 3 --> Tamaño en tanto por uno relativo al ancho de página
\usepackage{graphicx}

\newcommand{\imagen}[3]{
	\begin{figure}[!h]
		\centering
		\includegraphics[width=#3\textwidth]{#1}
		\caption{#2}\label{fig:#1}
	\end{figure}
	\FloatBarrier
}







\graphicspath{ {./img/} }

% Capítulos
\chapterstyle{bianchi}
\newcommand{\capitulo}[2]{
	\setcounter{chapter}{#1}
	\setcounter{section}{0}
	\setcounter{figure}{0}
	\setcounter{table}{0}
	\chapter*{#2}
	\addcontentsline{toc}{chapter}{#2}
	\markboth{#2}{#2}
}

% Apéndices
\renewcommand{\appendixname}{Apéndice}
\renewcommand*\cftappendixname{\appendixname}

\newcommand{\apendice}[1]{
	%\renewcommand{\thechapter}{A}
	\chapter{#1}
}

\renewcommand*\cftappendixname{\appendixname\ }

% Formato de portada

\makeatletter
\usepackage{xcolor}
\newcommand{\tutor}[1]{\def\@tutor{#1}}
%\newcommand{\tutorb}[1]{\def\@tutorb{#1}}

\newcommand{\course}[1]{\def\@course{#1}}
\definecolor{cpardoBox}{HTML}{E6E6FF}
\def\maketitle{
  \null
  \thispagestyle{empty}
  % Cabecera ----------------
\begin{center}
  \noindent\includegraphics[width=\textwidth]{cabeceraSalud}\vspace{1.5cm}%
\end{center}
  
  % Título proyecto y escudo salud ----------------
  \begin{center}
    \begin{minipage}[c][1.5cm][c]{.20\textwidth}
        \includegraphics[width=\textwidth]{escudoSalud.pdf}
    \end{minipage}
  \end{center}
  
  \begin{center}
    \colorbox{cpardoBox}{%
        \begin{minipage}{.8\textwidth}
          \vspace{.5cm}\Large
          \begin{center}
          \textbf{TFG del Grado en Ingeniería de la Salud}\vspace{.6cm}\\
          \textbf{\LARGE\@title{}}
          \end{center}
          \vspace{.2cm}
        \end{minipage}
    }%
  \end{center}
  
    % Datos de alumno, curso y tutores ------------------
  \begin{center}%
  {%
    \noindent\LARGE
    Presentado por \@author{}\\ 
    en Universidad de Burgos\\
    \vspace{0.5cm}
    \noindent\Large
    \@date{}\\
    \vspace{0.5cm}
    Tutor: \@tutor{}\\ 
    %Tutores: \@tutor{} -- \@tutorb{}\\
  }%
  \end{center}%
  \null
  \cleardoublepage
  }
\makeatother
\newcommand{\nombre}{Claudia Valentín Alguacil}
\newcommand{\nombreTutor}{Pedro Luis Sanchez Ortega} 
%\newcommand{\nombreTutorb}{Tutor 2} 
\newcommand{\dni}{03951611G} 

% Datos de portada
\title{Seguimiento clínico de pacientes con afectaciones en la mano con la ayuda de sensores de fuerza}
\author{\nombre}
\tutor{\nombreTutor}
\date{\today}

\begin{document}

\maketitle


\newpage\null\thispagestyle{empty}\newpage

%%%%%%%%%%%%%%%%%%%%%%%%%%%%%%%%%%%%%%%%%%%%%%%%%%%%%%%%%%%%%%%%%%%%%%%%%%%%%%%%%%%%%%%%
\thispagestyle{empty}


\noindent\includegraphics[width=\textwidth]{cabeceraSalud}\vspace{1cm}

\noindent D. \nombreTutor, profesor del departamento de Ingeniería Electromecánica, área Tecnología Electrónica.

\noindent Expone:

\noindent Que la alumna Dª. \nombre, con DNI \dni, ha realizado el Trabajo final de Grado en Ingeniería de la Salud titulado Seguimiento clínico de pacientes con afectaciones en la mano con la ayuda de sensores de fuerza. 

\noindent Y que dicho trabajo ha sido realizado por el alumno bajo la dirección del que suscribe, en virtud de lo cual se autoriza su presentación y defensa.

\begin{center} %\large
En Burgos, {\large \today}
\end{center}

\vfill\vfill\vfill

% Author and supervisor
\begin{minipage}{0.45\textwidth}
\begin{flushleft} %\large
Vº. Bº. del Tutor:\\[2cm]
D. \nombreTutor
\end{flushleft}
\end{minipage}
\hfill
\begin{minipage}{0.45\textwidth}
\begin{flushleft} %\large
%Vº. Bº. del Tutor:\\[2cm]
%D. \nombreTutorb
\end{flushleft}
\end{minipage}
\hfill

\vfill

% para casos con solo un tutor comentar lo anterior
% y descomentar lo siguiente
%Vº. Bº. del Tutor:\\[2cm]
%D. nombre tutor


\newpage\null\thispagestyle{empty}\newpage




\frontmatter

% Abstract en castellano
\renewcommand*\abstractname{Resumen}
\begin{abstract}
Este proyecto se centra en la búsqueda de resolver un problema real, creando un prototipo de dispositivo e interfaz para cuantificar la fuerza que los pacientes pueden ejercer en las sesiones de terapia y establecer un registro a lo largo del tiempo de estas.

A pesar de que estos tipos de dispositivos ya existen, los hospitales públicos no están dotados de ellos debido a su alto precio y los escasos recursos en algunas áreas de los hospitales. La iniciativa de la creación de este proyecto, surge como respuesta a la necesidad comentada desde el área de terapia ocupacional. 

Se han estudiado los dispositivos que ya están en el mercado, para posteriormente realizar una búsqueda de los componentes más idóneos para el dispositivo a crear. Se ha creado un prototipo físico con sensores de fuerza resistivos y una interfaz básica para la visualización y registro de las cuantificaciones de fuerza.
\end{abstract}

\renewcommand*\abstractname{Descriptores}
\begin{abstract}
Sensor de Fuerza Resistivo, terapia ocupacional, cuantificación, fuerza, dinamómetro , mano, Arduino, bajo coste.
\ldots
\end{abstract}

\clearpage

% Abstract en inglés
\renewcommand*\abstractname{Abstract}
\begin{abstract}
This project focuses on searching for a real issue by creating a prototype device and interface to quantify the strength that patients can apply during therapy sessions, and to establish a record of these measurements over time.

Despite similar devices already existing, many public hospitals often are not equipped with them due to their high cost and the limited resources available in some certain hospital departments. 
The initiative for this project arose in response to an actual need expressed by the occupational therapy department.

Research has been conducted on the devices currently available on the market, followed by an investigation on potential components that could be used to build a new device. A physical prototype has been developed using force sensing resistors, along with a basic interface for visualizing and recording the strength measurements.
\end{abstract}

\renewcommand*\abstractname{Keywords}
\begin{abstract}
Force sensing resistors, occupational therapy, quantification, strength, hand, dinamometer, Arduino, low cost.
\end{abstract}

\clearpage

% Indices
\tableofcontents

\clearpage

\listoffigures

\clearpage

\listoftables
\clearpage


\mainmatter
\capitulo{1}{Introducción}
Este proyecto surgió durante el desarrollo de las prácticas curriculares que realicé en diversas áreas del Hospital Universitario de Burgos, en particular en el área de Terapia Ocupacional. 

Durante este proceso pude observar la forma de trabajar de las terapeutas ocupacionales, los recursos que tienen y darme cuenta de las dificultades que enfrentan en su día a día, tanto por falta de recursos como por la falta de implementación tecnológica que les permitiría tanto optimizar los tratamientos como mejorar en el seguimiento de estos mismos.

Una de esas muchas necesidades que me comunicaron fue la de tener un dispositivo capaz de medir la fuerza ejercida por un paciente con la mano al realizar ejercicios. De tal manera que el dispositivo permitiría visualizar el valor de presión en el momento, sino también poder realizar un registro para que pudieran documentar estas mediciones para poder observar la evolución o no evolución de un paciente y/o ver el momento de la sesión en el que el paciente se fatiga.

El dispositivo no debe centrarse en una única patología de la mano, además tiene ser útil tanto para el tratamiento como para el seguimiento de las diferentes patologías de la mano.

La presente memoria del proyecto está estructurada en 8 capítulos, organizados especialmente para que faciliten su comprensión.
\begin{itemize}
    \item \textbf{Capítulo 1}: Descripción del contenido del proyecto, estructura de la memoria y materiales utilizados.
    \item \textbf{Capítulo 2}: Objetivos principales del proyecto y personales.
    \item \textbf{Capítulo 3}: Descripción de conceptos teóricos y estado del arte.
    \item \textbf{Capítulo 4}: Discusión clínica, exposición de la problemática y comparación con otros prototipos.
     \item \textbf{Capítulo 5}: Descripción de técnicas y herramientas utilizadas.
    \item \textbf{Capítulo 6}: Resumen de resultados.
    \item \textbf{Capítulo 7}: Conclusiones.
    \item \textbf{Capítulo 8}: Posibles líneas futuras que pueda tener el proyecto.
\end{itemize}
 
El conjunto de ficheros y desarrollo del proyecto están recopilados \href{https://github.com/cva1003/TFG}{aquí}. Para comprender mejor la estructura de los ficheros se recomienda revisar el Anexo C.
\include{./tex/2_objetivos}
\capitulo{3}{Conceptos teóricos}
\section{Conceptos teóricos básicos}
Para facilitar la comprensión del proyecto, he seleccionado una serie de conceptos útiles a definir. 
\begin{itemize}
    \item Dinamómetro: artefacto destinado a la medición de la fuerza y el peso de los objetos a partir de la elasticidad de un resorte o muelle elástico.[Significados,s.f]
    \item Inclinómetro:
    \item Terapia ocupacional: 

\end{itemize}

\section{Estado del arte.}

Para iniciar el desarrollo del proyecto, se realizó una búsqueda del panorama actual en la medida de fuerza. 

En esta subsección se analiza y sintetiza el conocimiento existente en el área específica del proyecto. El objetivo principal es situar el trabajo dentro de un marco teórico y tecnológico, analizando investigaciones previas, aportes y detectando problemas que se puedan resolver.
\subsection{Dinamómetros:}
En la actualidad, los dinamómetros son una de las herramientas más utilizadas  para la medida de fuerza de los pacientes durante las sesiones de terapia ocupacional, fisioterapia o medicina deportiva. Existen diferentes modelos que varían en precisión, tecnología, precio o aplicación. 
A continuación, he destacado algunos de ellos:
\begin{itemize}
    \item Activforce 2: Es un dispositivo portátil e inalámbrico creado por la empresa estadounidense Activforce, con sede en California. Combina un dinamómetro con un inclinómetro lo que permite medir la fuerza máxima,promedio, rango de movimiento y simetría derecha-izquierda tanto en fuerza como en movimiento.Todas las mediciones se registran en una aplicación compatible con Android e iOS, donde el usuario puede ver toda la información recabada por el dispositivo. En la \textit{Figura 3.1} se puede ver una imagen del producto.
    \begin{figure}[h]
        \centering
        \includegraphics[width=0.5\textwidth]{img/ActivForce_Device.jpg}
        \caption{Dispositivo ActivForce2}
        \label{fig:activforce}
    \end{figure}
    
    Además, el Activforce 2 incluye varios accesorios que facilitan su uso en diferentes partes del cuerpo, permitiendo la ejecución de una amplia variedad de ejercicios y evaluaciones. Es un dispositivo de compra libre cuyo precio ronda los 450€. En la \textit{Figura 3.2} se puede ver una imagen del producto y los accesorios.
    \begin{figure}[h]
        \centering
        \includegraphics[width=0.5\textwidth]{img/ActivForce_Attachments.jpg}
        \caption{Accesorios compatibles con el dispositivo ActivForce2}
        \label{fig:activforce}
    \end{figure}

    \item MicroFET 2:Es un dispositivo portátil e inalámbrico creado por la empresa belga MVS in motion. Es un dinamómetro diseñado para la evaluación y prueba de fuerza permite tomar mediciones de prueba muscular objetivas, cuantificables y confiables.
    Es utilizado para la ayuda en el diagnóstico, pronóstico y el tratamiento de trastornos musculares. Presenta una pantalla digital donde se puede observar el valor de las mediciones en tiempo real. 
    Es un dispositivo de compra libre cuyo precio ronda los 1200€. En la \textit{Figura 3.3} se puede ver una imagen del dispositivo.
    \begin{figure}[h]
        \centering
        \includegraphics[width=0.45\textwidth]{img/MicroFET 2.jpg}
        \caption{Dispositivo MicroFET 2}
        \label{fig:activforce}
    \end{figure}
    
    \item MAP 80K1S: Es un dispositivo portátil e inalámbrico creado por la empresa alemana KERN \& SOHN. Es un dinamómetro de mano utilizado específicamente para tratamientos de rehabilitación, para la determinación de la fuerza de cierre de mano.
    Presenta cuatro modos de medición: tiempo real, valor máximo. valor promedio y de contaje. 
    Es utilizado en sesiones de rehabilitación para detectar la disminución de fuerza y la evolución del paciente entre sesiones. Cuenta con una pantalla digital donde se puede observar las mediciones realizadas.
    Es un dispositivo de compra libre cuyo precio ronda los 280€. En la \textit{Figura 3.4} se puede ver una imagen del dispositivo.
      \begin{figure}[h]
        \centering
        \includegraphics[width=0.5\textwidth]{img/MAP-80K1S.jpg}
        \caption{Dispositivo MAP 80K1S}
        \label{fig:activforce}
    \end{figure}
    
    \item Squeezy dynamometer: Es un dispositivo portátil, un dinamómetro de presión hidráulica de mano cuya medición se realiza apretando la perilla, generando presión que se transfiere al calibrador, el cual muestra con precisión la fuerza ejercida. Es un dispositivo de compra libre cuyo precio ronda los 70 €. En la \textit{Figura 3.5} se puede ver una imagen del dispositivo.
    \begin{figure}[h]
        \centering
        \includegraphics[width=0.5\textwidth]{img/Dinamometro pera.jpeg}
        \caption{Dinamómetro de Pera}
        \label{fig:activforce}
    \end{figure}
\end{itemize}
\subsection{Artículos relacionados:}
En esta subsección presentaré los artículos que he encontrado relacionados con el área, abarcando tanto la creación de dispositivos para la medición de la fuerza como el uso de dinamómetros para el registro de la evolución de los pacientes.
\begin{itemize}
    \item \textbf{Dispositivo de medición de fuerza de los dedos y su
rol en el seguimiento de las funciones de la mano}

Juliana Gomez et al, en su articulo llamado 'Dispositivo de medición de fuerza de los dedos y su rol en el seguimiento de las funciones de la mano' publicado en la revista de cirugía plástica Ibero Latinoamericana aborda la creación de un dispositivo capaz de medir la fuerza de los dedos de manera individual, para su uso en la evaluación del paciente sano y de pacientes con patologías traumáticas y no traumáticas, para determinar grados de discapacidad, seguimiento de enfermedades y/o recuperaciones. 
Este dispositivo se basa en el uso de 5 sensores del tipo Flexiforce A301 de la empresa TESKA®, sistema embebido
tipo Arduino y Matlab, se puede observar en la \textit{Figura 3.6}. Además, afirman la validez de su uso tras realizar un estudio con veinte sujetos sanos, obteniendo tasas de medición equiparables a las de otros sistemas.
    \begin{figure}[h]
        \centering
        \includegraphics[width=0.7\textwidth]{img/dispositivo Revista.jpg}
        \caption{Dinamómetro creado por Juliana Gomez et al}
        \label{fig:activforce}
    \end{figure}
    \item \textbf{Uso del dinamómetro para mejorar la fuerza de la mano del adulto mayor.}
    Mayra Lasteña Millingalli Ortega et al, en su articulo de revisión llamado 'Uso del dinamómetro para mejorar la fuerza de la mano del adulto mayor' publicado en la Revista Científica Arbitrada Multidisciplinaria PENTACIENCIAS en Octubre de 2023, realizan un trabajo de investigación revisando artículos en ingles y español publicados entre 2013 y 2022.
    La población de estudio son pacientes geriátricos tratados en Terapia Ocupacional, las investigadoras exponen que actualmente en estas sesiones de terapia los pacientes son evaluados mediante métodos manuales como las escalas de registros, afirman que estos métodos son rudimentarios y no evidentes. 
\end{itemize} 

\capitulo{4}{Discusión clínica.}
\section{Justificación clínica del problema.}

\section{Implicaciones terapéuticas.}
Las implicaciones terapéuticas son los efectos y consecuencias que implican un tratamiento o dispositivo sobre un paciente, pueden ser positivas o negativas.

El uso del dispositivo puede ser interdisciplinar, ya sea durante sesiones de terapia ocupacional o fisioterapia o, incluso durante las consultas médicas de revisión. 

El uso principal previsto es la monitorización continua de los pacientes. Este dispositivo puede utilizarse en diversos momentos de las sesiones, ya sea solo con los sensores o bien con la adaptación de mangos u otros soportes adaptados.

El proceso es bastante sencillo, el profesional sanitario debe registrar a los pacientes que desea monitorizar e ir registrando diariamente la fuerza que son capaces de realizar los pacientes. Esto permitirá a los profesionales observar si existe una evolución favorable, negativa o estancada y, en consecuencia,decidir si las sesiones deberían continuar o cesar. 

Otro uso relevante es la detección de la fatiga muscular durante las sesiones. Lo que permitiría realizar los cambios oportunos para la optimización total de las sesiones, o bien reducir el tiempo. La fatiga muscular se detectaría si al realizar mediciones en varias ocasiones durante las sesiones,se registran picos de fuerza significativamente más bajos en comparación con las anteriores.

\section{Contexto de uso real.}
El dispositivo está diseñado para un perfil de pacientes amplio, en edad, patología o afectación.

En cuanto al rango de edad, el dispositivo puede utilizarse desde adolescentes hasta personas mayores. Aunque no se ha probado en niños, no se recomienda su uso , debido a que los sensores tienen un diámetro de 3 cm, lo que dificulta al menor la correcta disposición de sus dedos, debido al tamaño de sus manos.

La integración del dispositivo durante una sesión de terapia ocupacional o fisioterapia es bastante amplia.

En el área de terapia ocupacional, la Tabla Canadiense se utiliza como parte de las sesiones de toda patología que implique una afectación en mano. En este punto de la sesión sería un buen momento para introducir el dispositivo, colocando los sensores en el mango e introducirlo en una de las varas de la tabla canadiense. 

A continuación, el profesional debe realizar los siguientes pasos: 
\begin{enumerate}
    \item Abrir la interfaz, iniciar sesión y seleccionar o registrar al paciente.
    \item Seleccionar empezar a medir y expresar al paciente que debe ejercer fuerza durante el tiempo que la luz led este encendida.
    \item Visualización del nuevo registro y si se desea visualización de la gráfica. 
\end{enumerate}

En el área de fisioterapia, se podría realizar en cualquier parte de la sesión, ya sea solo o con algún soporte que se diseñe. Los pasos a realizar serían iguales que en los de una sesión de terapia ocupacional. 
\begin{enumerate}
    \item Abrir la interfaz, iniciar sesión y seleccionar o registrar al paciente.
    \item Seleccionar empezar a medir y expresar al paciente que debe ejercer fuerza durante el tiempo que la luz led esté encendida.
    \item Visualización del nuevo registro y si se desea visualización de la gráfica. 
\end{enumerate}

Además, se podría utilizar durante las revisiones médicas, durante el reconocimiento. Pudiendo el médico visualizar si de una revisión a otra existen cambios. 

\section{ Comparativa con estudios o protocolos clínicos existentes.}

A partir de los estudios revisados en la sección 'Estado del arte' del apartado 'Conceptos teóricos', se va a comparar los estudios, prototipos o protocolos existentes con la propuesta presentada con este proyecto, destacando ventajas, limitaciones y diferencias.

En los diseños analizados (Juliana Gomez et al, Luis Carlos Ralon Gordillo y el mio proprio) , existe un objetivo común de la monitorización de la fuerza ejercida por pacientes en terapia de mano. 

Aunque en los 3 se utiliza el sensor de fuerza resistivos(FSR) como elemento principal, tanto en los dispositivos de Juliana Gomez et al y Luis Carlos Ralon se implementa el uso de resortes.

En la tabla \ref{tab:comparativa_prototipos} se presenta una tabla comparativa de los 3 dispositivos.
\begin{table}[h]
    \begin{tabular}{|p{2cm}|p{4cm}|p{4cm}|p{4cm}|}
    \hline
    \rowcolor[HTML]{BFBFBF} 
    \textbf{Aspecto} & \textbf{Juliana Gómez et al.} & \textbf{Luis Carlos Ralón Gordill}& \textbf{Diseño propio} \\ \hline
    Sensor utilizado & FSR & FSR & FSR \\ \hline
    Número de sensores & 5 & 1 & 5 \\ \hline
    Uso del resorte & Transmisión de fuerza & Generar resistencia & No aplica \\ \hline
    Mecanismo de transmisión & Mediante resorte & Mediante resorte& Directo \\ \hline
    Ventajas & Alta precisión & Uso en niños & Uso aplicabilidad múltiple\\ \hline
    Limitaciones & Solo permite el agarre de precisión & Fallos en Resorte-Sensor & . \\ \hline
    Coste  & No menciona & No menciona precio exacto, pero deja claro que busca producir el menos coste y al utilizar un único sensor el coste total es bajo. & Menor posible,  \\ \hline
    \end{tabular}
    \caption{Comparativa de prototipos}
    \label{tab:comparativa_prototipos}
\end{table}

\section{Limitaciones y riesgos clínicos.}
Este dispositivo presenta un perfil de seguridad óptimo para su uso clínico, no supone ningún riesgo para pacientes o profesionales sanitarios. Los sensores que incorpora, son sensores resistentes al agua, que permite una desinfección rápida, óptima y efectiva entre pacientes mediante el uso de toallitas o paños húmedos con desinfectante. Siempre evitando que se sumerjan completamente en cualquier solución líquida, debido a que son componentes eléctricos.

Si se añaden accesorios:
\begin{itemize}
    \item Mango impresión 3D: su desinfección depende mucho del material con el que se ha impreso. Si utilizamos PLA (material pensado debido a su bajo coste), no debemos utilizar desinfectantes que contengan acetonas ya que este producto es capaz de degradarlo. Se podrían usar toallitas antibacterianas o paños humedecidos en soluciones desinfectantes.
    \item Otros soportes: se deben desinfectar según las especificaciones del material del que estén fabricados para no degradarlo.  
\end{itemize}

Un protocolo de limpieza adecuado garantiza las condiciones higiénicas necesarias para un uso seguro y continuado sin comprometer la salud de los pacientes o de los sanitarios que lo manipulen.
\include{./tex/5_metodologia}
\capitulo{6}{Resultados}

\section{Resumen de resultados}

Se presenta en este capítulo un resumen de los resultados obtenidos durante el desarrollo del proyecto. 
\begin{itemize}
    
    \item \textbf{Elección de un sensor capaz de medir la fuerza ejercida por cada dedo.}
    
    Se ha utilizado el sensor de fuerza resistivo MD30-60, ofreciendo resultados satisfactorios en las pruebas realizadas. Con el objetivo de lograr un ajuste de las medidas lo más cercano a la realidad, se realizó una calibración en los sensores, para más detalle véase el Apendice G.1 del anexo. Además se realizó un análisis del porcentaje de error, ofreciendo como resultado, un valor de error medio de 4.6\%, con valores de pico de hasta un 10\% error.
    
    \item \textbf{Creación de un prototipo de dispositivo que pueda medir la presión ejercida por la mano.}
    
    Se ha logrado la creación de un prototipo funcional del dispositivo (\ref{fig:Prototipo Fisico}), cumpliendo con el diseño inicial de medir la presión ejercida por cada dedo de la mano de forma orientativa y repetible. Este prototipo integra cinco sensores de fuerza resistivos, una placa de arduino, una placa de pruebas, una luz led, un pin botón y otros materiales complementarios como resistencias y cables, proporcionando un prototipo integral que cumple con los requisitos necesarios. 

    Adicionalmente, se ha diseñado un accesorio complementario (\ref{fig:Accesorio}) que facilita el uso del dispositivo durante sesiones de rehabilitación. Tiene una dimensión de 60x60x135, con un agujero interior para insertar una de las barras de la
    tabla canadiense y cinco hendiduras destinadas a situar cada uno de los sensores.

    \begin{figure}
        \centering
        \includegraphics[width=0.25\linewidth]{img/Mango3.png}
        \caption{Accesorio para tabla canadiense. Fuente propia}
        \label{fig:Accesorio}
    \end{figure}
    \begin{figure}
        \centering
        \includegraphics[angle=180,width=0.5\linewidth]{img/Prototipo_s.png}
        \caption{Prototipo físico. Fuente propia}
        \label{fig:Prototipo Fisico}
    \end{figure}
    \item \textbf{Seguimiento de evolución de los pacientes.}

    Se ha desarrollado un prototipo de interfaz tipo desplegable, capaz de almacenar los datos obtenidos por los sensores en un archivo xlsx. Además de su almacenamiento, permite al profesional la visualización de estos mediante una tabla (\ref{fig:tabla-registro}) o representando los datos mediante una gráfica (\ref{fig:Grafica-datos}). 
    
    La estructura y visualización de los datos recolectados se amplía en el Apéndice D del anexo.
    \begin{figure}
        \centering
        \includegraphics[width=0.8\linewidth]{img/grafica_fuerza.png}
        \caption{Representación datos en gráfica. Fuente propia}
        \label{fig:Grafica-datos}
    \end{figure}
    \begin{figure}
        \centering
        \includegraphics[width=1\linewidth]{img/Datos_medidas.png}
        \caption{Tabla de registros. Fuente propia}
        \label{fig:tabla-registro}
    \end{figure}
    \item \textbf{Comparativa de prototipos}
    
    Tras la realización del proyecto, he realizado una comparativa con otros prototipos mencionados en la sección 'Estado del arte'.

    A pesar de que los 3 prototipos emplean los sensores resistivos y tienen como finalidad el seguimiento de los pacientes para facilitar la recuperación, existen algunas diferencias. 
    En la tabla \ref{tab:comparacion_diseños} se presenta una comparativa de los 3 dispositivos.
\end{itemize}


\begin{table}[h]
    \centering
    \begin{tabular}{|p{2.5cm}|p{3.5cm}|p{4cm}|p{3.8cm}|}
    \rowcolor[HTML]{BFBFBF} 
    \hline
    \textbf{Aspecto} & \textbf{Juliana Gómez et al.} & \textbf{Luis Carlos Ralón Gordill} & \textbf{Diseño propio} \\ \hline
    Sensor utilizado   & FSR  & FSR  & FSR \\ \hline
    Número de sensores   & 5  & 1  & 5 \\ \hline
    Uso del resorte & Transmisión de fuerza & Generar resistencia  & No aplica   \\ \hline
    Mecanismo de transmisión   & Mediante resorte  & Mediante resorte & Directo  \\ \hline
    Ventajas   & Alta precisión  & Uso en niños  & Uso aplicabilidad múltiple \\ \hline
    Limitaciones & Solo permite el agarre de precisión & Fallos en Resorte-Sensor & Error medio de 4.6\% \\ \hline
    Visualización & Interfaz, visualización en el momento & Interfaz gráfica, visualización en el momento& Interfaz desplegable, con posibilidad de ver registros anteriores \\ \hline
    Coste    & No menciona    & No menciona precio exacto, pero deja claro que busca producir el menos coste y al utilizar un único sensor el coste total es bajo. & 78€ \\ \hline
    \end{tabular}
    \caption{Comparación de los prototipos.}
    \label{tab:comparacion_diseños}
\end{table}
    
\section{Discusión}

Los resultados obtenidos cumplen con los objetivos marcados en el inicio del proyecto. Se ha logrado realizar un prototipo de interfaz y dispositivo capaz de registrar los datos recogidos por los sensores.

La implementación de los gráficos en la visualización de los registros permite una representación clara de la evolución de los pacientes durante las sesiones. Facilitando la interpretación de los datos recogidos por los profesionales sanitarios, detectando progresos, fatiga muscular o estancamientos, factores muy útiles en la toma de decisiones médicas.

Además, la protección de los datos es uno de los aspectos fundamentales en la creación del dispositivo. Se cumplen en todo momento mediante la encriptación de los documentos. El uso de bibliotecas para el cifrado se detalla en el Apéndice C.1 y D.1 del anexo.

Por último, destacar la utilización de materiales que impliquen el menor coste posible en la realización del dispositivo. El análisis económico detallado y la viabilidad legal se exponen en los Apéndices A.3 y A.4 del anexo.
\capitulo{7}{Conclusiones}

Concluido el desarrollo integral del proyecto, se presentan las principales conclusiones obtenidas del trabajo realizado:

\begin{itemize}
    \item El sensor de fuerza resistivo seleccionado cumple con los requisitos esperables, siendo favorable su utilización como método de cuantificación, con un promedio de error de 4.6\%.
    \item El diseño de los prototipos, tanto físico como de software, cumplen con los requisitos planteados. 
    \item Actualmente, no existe un dispositivo de bajo coste en el mercado que realice esta tarea, por lo que mejorar el diseño propuesto puede abrir una oportunidad dentro del mercado tecnológico-sanitario.
\end{itemize}
\capitulo{8}{Líneas Futuras}
Al tratarse de un prototipo, se pueden realizar varias mejoras para llegar a un dispositivo comerciable , robusto y de bajo coste.

Una de las primeras mejoras que podría implementarse es la eliminación del botón físico del dispositivo destinado a inicio de la medición, de manera que esta se active directamente al presionar 'Iniciar medición' en la interfaz. 

Asimismo, se podrían diseñar diferentes tipos de mangos donde poner los sensores, variando tanto en tamaño como en los materiales a utilizar. Debido a que el medio utilizado es impresión 3D, las posibilidades de personalización son muy amplias. Se podría utilizar material semielástico, permitiendo así una leve deformación, ya que estos materiales combinan flexibilidad y resistencia. Otra posibilidad, podría ser emplear compuestos como mezclas de PLA con polvo metálico, que aportarían mayor rigidez y peso, lo que supondría un desafío adicional durante el ejercicio.

De otra manera, se puede emplear un cambio en el tamaño ajustando el diámetro del mango. Contar con estos diferentes diámetros facilita el agarre para las personas según las diferentes patologías que tengan ya sea que afecten la movilidad o fuerza de la mano.

Además, se podría implementar una funcionalidad que permita generar un informe final por paciente imprimible, permitiendo que forme parte de la historia clínica. Usando la biblioteca FPDF, podemos generar archivos pdf con posibilidad de imprimir.

Por último, mejorar el desplegable para que se abra como una aplicación más de escritorio y no haga falta acceder desde el CMD o terceras aplicaciones como Visual Studio Code.
Con kivymd solo se podrá realizar esto para archivos con sistema operativo windows, para realizarlo se deberá crear un paquete especial siguiendo los pasos de la documentación de Kivymd \cite{kivymdapp}

\bibliographystyle{apalike}
\bibliography{bibliografia}

\end{document}