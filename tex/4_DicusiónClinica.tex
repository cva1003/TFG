\capitulo{4}{Discusión clínica.}
\section{Justificación clínica del problema.}

\section{Implicaciones terapéuticas.}
Las implicaciones terapéuticas son los efectos y consecuencias que implican un tratamiento o dispositivo sobre un paciente, pueden ser positivas o negativas.

El uso del dispositivo puede ser interdisciplinar, ya sea durante sesiones de terapia ocupacional o fisioterapia o, incluso durante las consultas médicas de revisión. 

El uso principal previsto es la monitorización continua de los pacientes. Este dispositivo puede utilizarse en diversos momentos de las sesiones, ya sea solo con los sensores o bien con la adaptación de mangos u otros soportes adaptados.

El proceso es bastante sencillo, el profesional sanitario debe registrar a los pacientes que desea monitorizar e ir registrando diariamente la fuerza que son capaces de realizar los pacientes. Esto permitirá a los profesionales observar si existe una evolución favorable, negativa o estancada y, en consecuencia,decidir si las sesiones deberían continuar o cesar. 

Otro uso relevante es la detección de la fatiga muscular durante las sesiones. Lo que permitiría realizar los cambios oportunos para la optimización total de las sesiones, o bien reducir el tiempo. La fatiga muscular se detectaría si al realizar mediciones en varias ocasiones durante las sesiones,se registran picos de fuerza significativamente más bajos en comparación con las anteriores.

\section{Contexto de uso real.}


\section{ Comparativa con estudios o protocolos clínicos existentes.}
\section{Limitaciones y riesgos clínicos.}