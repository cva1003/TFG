\capitulo{1}{Introducción}
Este proyecto surgió durante el desarrollo de las prácticas curriculares que realicé en diversas áreas del Hospital Universitario de Burgos, en particular en el área de Terapia Ocupacional. 

Durante este proceso pude observar su forma de trabajar, los recursos que tienen y darme cuenta de las dificultades que enfrentan en su día a día, tanto por falta de recursos como por la falta de implementación tecnológica que les permitiría tanto optimizar los tratamientos como mejorar en el seguimiento de estos mismos.

Una de esas muchas necesidades que me comunicaron fue la de tener un dispositivo capaz de medir la fuerza ejercida por un paciente con la mano al realizar ejercicios. De tal manera que el dispositivo permitiría visualizar el valor de presión en el momento, sino también poder realizar un registro para que pudieran documentar estas mediciones para poder observar la evolución o no evolución de un paciente y/o ver el momento de la sesión en el que el paciente se fatiga.

El dispositivo no debe centrarse en una única patología de la mano, además debe ser útil tanto para el tratamiento como para el seguimiento de las diferentes patologías de la mano.

La presente memoria del proyecto está estructurada en 8 capítulos, organizados especialmente para que faciliten su comprensión.
\begin{itemize}
    \item \textbf{Capítulo 1}: Descripción del contenido del proyecto, estructura de la memoria y materiales utilizados.
    \item \textbf{Capítulo 2}: Objetivos principales del proyecto y personales.
    \item \textbf{Capítulo 3}: Descripción de conceptos teóricos y estado del arte.
    \item \textbf{Capítulo 4}: Discusión clínica, exposición de la problemática y comparación con otros prototipos.
     \item \textbf{Capítulo 5}: Descripción de técnicas y herramientas utilizadas.
    \item \textbf{Capítulo 6}: Resumen de resultados.
    \item \textbf{Capítulo 7}: Conclusiones.
    \item \textbf{Capítulo 8}: Posibles líneas futuras que pueda tener el proyecto.
\end{itemize}