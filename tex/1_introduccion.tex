\capitulo{1}{Introducción} 
Este proyecto tuvo su origen durante las prácticas curriculares que realice en diversas áreas del Hospital Universitario de Burgos, en particular en el área de Terapia Ocupacional. 
Las terapeutas ocupacionales me dieron a conocer su trabajo, incluso en algunas ocasiones haciéndome partícipe de él. Durante este proceso, pude observar su forma trabajar, los recursos que tienen y las diferentes dificultades que enfrentan en su día a día, tanto por falta de recursos como por la falta implementación tecnológica que les permitiría tanto a optimizar los tratamientos como una mejora en el seguimiento de estos mismos. 
Una de las principales necesidades identificadas fue la de tener un dispositivo capaz de medir la presión ejercida por un paciente con la mano al realizar ejercicios con la tabla canadiense, herramienta principal que utilizan para el tratamiento en las patologías de mano.Este dispositivo no solo debería permitir visualizar el valor de presión en el momento, sino también poder realizar un registro con el cual tener un seguimiento de estas mediciones para poder observar la evolución o no evolución de un paciente y/o ver el momento de la sesión en el que el paciente se fatiga. 
