\capitulo{1}{Introducción}
Este proyecto tuvo su origen durante las prácticas curriculares que realicé en diversas áreas del Hospital Universitario de Burgos, en particular en el área de Terapia Ocupacional. 

Las terapeutas ocupacionales me dieron a conocer su trabajo, incluso en algunas ocasiones haciéndome partícipe de ello. Durante este proceso pude observar su forma de trabajar, los recursos que tienen y las diferentes dificultades que enfrentan en su día a día, tanto por falta de recursos como por la falta de implementación tecnológica que les permitiría tanto optimizar los tratamientos como mejorar en el seguimiento de estos mismos.

Una de las muchas necesidades que me comunicaron fue la de tener un dispositivo capaz de medir la presión ejercida por un paciente con la mano al realizar ejercicios con la tabla canadiense. De tal manera que el dispositivo permitiría visualizar el valor de presión en el momento, sino también poder realizar un registro pudieran tener un registro de estas mediciones para poder observar la evolución o no evolución de un paciente y/o ver el momento de la sesión en el que el paciente se fatiga.

No está pensado para enfocarse en una única patología de la mano, sino que pretende ser útil tanto para el tratamiento como para el seguimiento de diferentes patologías de la mano.

La presente memoria del proyecto está estructurada en 7 capítulos, organizados especialmente para que faciliten su comprensión.
\begin{itemize}
    \item \textbf{Capítulo 1}: Descripción del contenido del proyecto,estructura de la memoria y materiales utilizados.
    \item \textbf{Capítulo 2}: Objetivos principales del proyecto y personales.
    \item \textbf{Capítulo 3}:Descripción de conceptos teóricos y estado del arte.
    \item \textbf{Capítulo 4}:Descripción de técnicas y herramientas utilizadas
    \item \textbf{Capítulo 5}:Resumen de resultados 
    \item \textbf{Capítulo 6}:Conclusiones 
    \item \textbf{Capítulo 7}:Posibles lineas futuras que pueda tener el proyecto
\end{itemize}