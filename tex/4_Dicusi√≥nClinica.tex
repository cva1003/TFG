\capitulo{4}{Discusión clínica.}
\section{Justificación clínica del problema.}

\section{Implicaciones terapéuticas.}
Las implicaciones terapéuticas son los efectos y consecuencias que implican un tratamiento o dispositivo sobre un paciente, pueden ser positivas o negativas.

El uso del dispositivo puede ser interdisciplinar, ya sea durante sesiones de terapia ocupacional o fisioterapia o, incluso durante las consultas médicas de revisión. 

El uso principal previsto es la monitorización continua de los pacientes. Este dispositivo puede utilizarse en diversos momentos de las sesiones, ya sea solo con los sensores o bien con la adaptación de mangos u otros soportes adaptados.

El proceso es bastante sencillo, el profesional sanitario debe registrar a los pacientes que desea monitorizar e ir registrando diariamente la fuerza que son capaces de realizar los pacientes. Esto permitirá a los profesionales observar si existe una evolución favorable, negativa o estancada y, en consecuencia,decidir si las sesiones deberían continuar o cesar. 

Otro uso relevante es la detección de la fatiga muscular durante las sesiones. Lo que permitiría realizar los cambios oportunos para la optimización total de las sesiones, o bien reducir el tiempo. La fatiga muscular se detectaría si al realizar mediciones en varias ocasiones durante las sesiones,se registran picos de fuerza significativamente más bajos en comparación con las anteriores.

\section{Contexto de uso real.}
El dispositivo está diseñado para un perfil de pacientes amplio, en edad, patología o afectación.

En cuanto al rango de edad, el dispositivo puede utilizarse desde adolescentes hasta personas mayores. Aunque no se ha probado en niños, no se recomienda su uso , debido a que los sensores tienen un diámetro de 3 cm, lo que dificulta al menor la correcta disposición de sus dedos, debido al tamaño de sus manos.

La integración del dispositivo durante una sesión de terapia ocupacional o fisioterapia es bastante amplia.

En el área de terapia ocupacional, la Tabla Canadiense se utiliza como parte de las sesiones de toda patología que implique una afectación en mano. En este punto de la sesión sería un buen momento para introducir el dispositivo, colocando los sensores en el mango e introducirlo en una de las varas de la tabla canadiense. 

A continuación, el profesional debe realizar los siguientes pasos: 
\begin{enumerate}
    \item Abrir la interfaz, iniciar sesión y seleccionar o registrar al paciente.
    \item Seleccionar empezar a medir y expresar al paciente que debe ejercer fuerza durante el tiempo que la luz led este encendida.
    \item Visualización del nuevo registro y si se desea visualización de la gráfica. 
\end{enumerate}

En el área de fisioterapia, se podría realizar en cualquier parte de la sesión, ya sea solo o con algún soporte que se diseñe. Los pasos a realizar serían iguales que en los de una sesión de terapia ocupacional. 
\begin{enumerate}
    \item Abrir la interfaz, iniciar sesión y seleccionar o registrar al paciente.
    \item Seleccionar empezar a medir y expresar al paciente que debe ejercer fuerza durante el tiempo que la luz led esté encendida.
    \item Visualización del nuevo registro y si se desea visualización de la gráfica. 
\end{enumerate}

Además, se podría utilizar durante las revisiones médicas, durante el reconocimiento. Pudiendo el médico visualizar si de una revisión a otra existen cambios. 

\section{ Comparativa con estudios o protocolos clínicos existentes.}

A partir de los estudios revisados en la sección 'Estado del arte' del apartado 'Conceptos teóricos', se va a comparar los estudios, prototipos o protocolos existentes con la propuesta presentada con este proyecto, destacando ventajas, limitaciones y diferencias.

En los diseños analizados (Juliana Gomez et al, Luis Carlos Ralon Gordillo y el mio proprio) , existe un objetivo común de la monitorización de la fuerza ejercida por pacientes en terapia de mano. 

Aunque en los 3 se utiliza el sensor de fuerza resistivos(FSR) como elemento principal, tanto en los dispositivos de Juliana Gomez et al y Luis Carlos Ralon se implementa el uso de resortes.

En la tabla \ref{tab:comparativa_prototipos} se presenta una tabla comparativa de los 3 dispositivos.
\begin{table}[h]
    \begin{tabular}{|p{2cm}|p{4cm}|p{4cm}|p{4cm}|}
    \hline
    \rowcolor[HTML]{BFBFBF} 
    \textbf{Aspecto} & \textbf{Juliana Gómez et al.} & \textbf{Luis Carlos Ralón Gordill}& \textbf{Diseño propio} \\ \hline
    Sensor utilizado & FSR & FSR & FSR \\ \hline
    Número de sensores & 5 & 1 & 5 \\ \hline
    Uso del resorte & Transmisión de fuerza & Generar resistencia & No aplica \\ \hline
    Mecanismo de transmisión & Mediante resorte & Mediante resorte& Directo \\ \hline
    Ventajas & Alta precisión & Uso en niños & Uso aplicabilidad múltiple\\ \hline
    Limitaciones & Solo permite el agarre de precisión & Fallos en Resorte-Sensor & . \\ \hline
    Coste  & No menciona & No menciona precio exacto, pero deja claro que busca producir el menos coste y al utilizar un único sensor el coste total es bajo. & Menor posible,  \\ \hline
    \end{tabular}
    \caption{Comparativa de prototipos}
    \label{tab:comparativa_prototipos}
\end{table}

\section{Limitaciones y riesgos clínicos.}
Este dispositivo presenta un perfil de seguridad óptimo para su uso clínico, no supone ningún riesgo para pacientes o profesionales sanitarios. Los sensores que incorpora, son sensores resistentes al agua, que permite una desinfección rápida, óptima y efectiva entre pacientes mediante el uso de toallitas o paños húmedos con desinfectante. Siempre evitando que se sumerjan completamente en cualquier solución líquida, debido a que son componentes eléctricos.

Si se añaden accesorios:
\begin{itemize}
    \item Mango impresión 3D: su desinfección depende mucho del material con el que se ha impreso. Si utilizamos PLA (material pensado debido a su bajo coste), no debemos utilizar desinfectantes que contengan acetonas ya que este producto es capaz de degradarlo. Se podrían usar toallitas antibacterianas o paños humedecidos en soluciones desinfectantes.
    \item Otros soportes: se deben desinfectar según las especificaciones del material del que estén fabricados para no degradarlo.  
\end{itemize}

Un protocolo de limpieza adecuado garantiza las condiciones higiénicas necesarias para un uso seguro y continuado sin comprometer la salud de los pacientes o de los sanitarios que lo manipulen.