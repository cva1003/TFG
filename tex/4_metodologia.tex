\capitulo{4}{Metodología}
En este capítulo voy a desarrollar los datos, las técnicas y métodos utilizados durante el proyecto. 
\section{Descripción de los Datos}
Para la realización del proyecto, no se han utilizado datos predefinidos por bases de datos, APIs públicas u otras fuentes de información.Los datos que aparecen son los obtenidos por los sensores durante las pruebas del dispositivo.
\section{Técnicas metodológicas de programación}

\section{Entornos y aplicaciones}
En esta sección voy a realizar una recopilación de aplicaciones y entornos que he utilizado para la realización del proyecto, todas ellas explicadas y utilizadas en diferentes asignaturas del grado.
\subsection{Overleaf}
Herramienta en línea que permite redactar,editar,exportar y compartir con otros usuarios documentos científicos en formato LaTeX.
He utilizado esta herramienta en la creación de este documento.
\subsection{R}
R es un entorno de software libre para computación estadística y gráficos. Tiene la capacidad de ejecutarse y compilarse en una amplia variedad de plataformas como Linux,MAC y Windows.

En este proyecto, se utiliza R para realizar una unión entre el repositorio de GitHub y la carpeta de mi escritorio del ordenador donde voy avanzando el proyecto.
\subsection{GitHub}
GitHub es una plataforma basada en la nube cuya finalidad es almacenar,compartir y trabajar en conjunto con otros usuarios. 
Además, permite tener un seguimiento del trabajo lo que permite ver todas las versiones que se van realizando de este mismo. 

En este proyecto, GitHub se ha utilizado para realizar un seguimiento del proyecto, permitiendo que tanto el tutor como los miembros del tribunal puedan visualizar los cambios realizados y participar activamente en el proceso.
\subsection{Tinkercad}
Tinkercad es una aplicación web que permite la realización de diseños en 3D, la realización de circuitos y la posibilidad de escribir programas para dar vida a los diseños mediante la codificación.

En este proyecto, ha sido utilizada para la realización de la simulación del diseño del circuito electrónico ya que permite utilizar diseños electrónicos ya hechos, modificarlos, añadir más componentes o empezar de 0 un circuito.
