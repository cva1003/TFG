\apendice{Manual del  programador.} 
\section{Estructura de directorios}
A continuación, se va a explicar la estructura de directorios situados en el Github.
\begin{itemize}
    \item AplicaciónEscritorio: Carpeta que contiene todo el código para la implementación de la interfaz.
    \begin{itemize}
        \item app.py: código de la interfaz
        \item interfaz.kv: código con el que se ha diseñado la interfaz
        \item Arduino.ino: Código arduino con el que se pueden realizar mediciones sin necesidad de la interfaz
    \end{itemize}
    \item Diseño: Carpeta que contiene el diseño del mango
    \item img: Carpeta que contiene todas las imágenes que se han utilizado para la realización del proyecto 
    \item tex:Carpeta que contiene todos los capítulos de la memoria y anexos
    \begin{itemize}
        \item 1\_introducción.tex:
        \item 2\_objetivos.tex:
        \item 3\_teoricos.tex:
        \item 4\_metodología.tex:
        \item 5\_resultados.tex:
        \item 6\_conclusión.tex:
        \item 7\_lineas\_futuras.tex:
        \item A\_planificación.tex:
        \item B\_manual\_usuario.tex:
        \item C\_manual\_programador.tex:
        \item D\_datos.tex:
        \item E\_diseño.tex:
        \item F\_requisitos.tex:
        \item G\_experimental.tex:
        \item H\_ODS.tex:
    \end{itemize}
    \item Readme.md
    \item Memoria\_Anexos\_PDF: Carpeta que contiene el documento de memoria y el de anexo en formato pdf.
    \begin{itemize}
        \item memoria.pdf: documento pdf con la memoria completa.
        \item anexo.pdf: documento pdf con el anexo completo.
    \end{itemize}
    \item anexo.tex:Documento latex que contiene la estructura del anexo.
    \item memoria.tex: Documento latex que contiene la estructura de la memoria.
    \item bibliografia.bib:Documento que recoge toda la bibliografía empleada en la memoria
    \item bibliografiaAnexos.bib:Documento que recoge toda la bibliografía empleada en el anexo
\end{itemize}
\section{Compilación, instalación y ejecución del proyecto}


\section{Instrucciones para la modificación o mejora del proyecto.}
