\capitulo{1}{Planificación}
\section{Introducción}
Este anexo tiene como objetivo una presentación de manera integral de todos los aspectos clave relacionados con el desarrollo y la implementación del proyecto. 

Para realizar un correcto desarrollo del proyecto, se ha definido una planificación temporal y se desarrollará una planificación económica para saber el coste aproximado total del producto desarrollado.
\section{Planificación temporal}
El proyecto se ha desarrollado a lo largo de un total de 14 semanas(aproximadamente unos 4 meses), durante las cuales se ha distribuido el trabajo de forma progresiva. 

En la \textit{tabla \ref{fig:Planificación}} se puede observar una tabla donde se ha recogido la planificación temporal.
\begin{figure}[h]
        \centering
        \includegraphics[,width=0.5\textwidth]{img/planificacion.png}
        \caption{Planificación del proyecto}
        \label{fig:Planificación}
    \end{figure}

\section{Planificación económica}

La planificación económica consiste en el análisis total del precio del proyecto, incluyendo de manera individual los precios de los componentes del dispositivo (hardware), software y personal.

Todos los precios han sido obtenidos en 2025, por lo cual el coste puede variar en un futuro. 
\subsection{Precios Hardware}
Para calcular el precio total de los costes de los materiales del dispositivo, se tiene en cuenta la suma total de los precios de los componentes así como una suma de los gastos de producción y un porcentaje destinado a los beneficios. 

En la tabla \ref{tab:costes_hardware} se ha realizado un resumen de gastos y precio total. 
\begin{table}[]
\centering
\begin{tabular}{|l|p{8cm}|l|}
\hline
\rowcolor[HTML]{BFBFBF} 
\textbf{} & \textbf{Cálculos} & \textbf{Precio} \\ \hline
Gastos de los componentes & Suma total de los precios de los componentes del dispositvo & 60,37€\\ \hline
Gastos de producción & 10\% del precio de los
 componentes & 6,04€\\ \hline
Ingresos destinados a beneficio & 15\% de los gastos totales & 9,96€\\ \hline
\textbf{Total}& Suma total de los gastos & 76,37€ \\ \hline
\end{tabular}
\caption{Costes de dispositivos hardware.}
\label{tab:costes_hardware}
\end{table}

\subsubsection{\textbf{Desglose de precios de los componentes}}
Los precios de cada componente han sido seleccionados, siendo los mínimos encontrados en el mercado de páginas oficiales.
\begin{itemize}
    \item Placa Elegoo+USB: 15,99€
    \item Resistencias 5*0,01299€=0.064€
    \item Sensores de fuerza: 5*8,1€= 40,5€
    \item Elementos conectores(cables dupont): 14*0,07€=0.98€
    \item Protoboard=2,83€
\end{itemize}
No se han contemplado los gastos de la compra de un ordenador, ya que en la actualidad en toda área del hospital se cuenta con uno de forma habitual. 
\subsection{Precios software}

No se incluye ningún gasto en software ya que todo el material utilizado es gratuito y de código abierto.

\subsection{Precio de personal}
Para calcular el salario del personal se ha partido de los datos publicados de salarios de ingenieros biomédicos, profesionales con un perfil técnico y formativo parecido al ingeniero de la salud.
Actualmente, en España el sueldo medio de un ingeniero biomédico es de 3.500€ \cite{SueldoBioing}\footnote{Pagina web de indeed, donde se puede ver sueldos de diferentes trabajos  \cite{SueldoBioing}.}\cite{SUELDO}. Este sueldo es más bajo en recién egresados suponiendo un sueldo aproximado entre 1.500€ y 1.900€ mensuales.\cite{Sueldo_egresado}\footnote{Pagina web de la UAX con información económica sobre el salario de un graduado en ingeniería biomédica? \cite{Sueldo_egresado}.}.


En la tabla \ref{tab:costes_personal} se recoge el sueldo aproximado que debería cobrar por el proyecto un ingeniero de la salud.

\begin{table}[]
\centering
\begin{tabular}{|l|l|}
\hline
\rowcolor[HTML]{BFBFBF} 
\textbf{} & \textbf{Precio} \\ \hline
Sueldo mensual & 1.600€ \\ \hline
\textbf{Total en 4 meses }& 6.400€ \\ \hline
\end{tabular}
\caption{Costes de sueldos del personal.}
\label{tab:costes_personal}
\end{table}

\begin{table}[]
\centering
\begin{tabular}{|l|l|}
\hline
\rowcolor[HTML]{BFBFBF} 
\textbf{} & \textbf{Precio} \\ \hline
Material & 76,37€ \\ \hline
Sueldo  &  6.400€ \\ \hline
\textbf{Total }& 6.476,38€ \\ \hline
\end{tabular}
\caption{Precio total del proyecto}
\label{tab:Costes Totales}
\end{table}

\section{Viabilidad legal}
Se debe tener en cuenta la normativa legal desde la creación de la idea hasta la comercialización y uso postventa.

Es esencial la implementación de leyes que garanticen el cumplimiento de la normativa aplicable que respalde los derechos y obligaciones del autor.Incluyendo desde leyes de las regulaciones técnicas, de propiedad intelectual y de responsabilidad profesional, conforme a lo establecido en la legislación.

Asimismo, se debe implementar medidas vigorosas que protejan la integridad de los usuarios y la confidencialidad de sus datos. 

Podemos dividir el proceso en 2 fases, la primera fase incluiría la creación de la idea, diseño,desarrollo y realización de pruebas, y una segunda fase que incluya la comercialización y postventa.

\subsection{1º Fase}

Durante la primera fase de creación de la idea, diseño, desarrollo y realización de pruebas, se deberán incluir las siguientes normas legislativas: 

\begin{itemize}
    \item Ley 24/2015, de 24 de julio \cite{boe--2015-8328}, de Patentes: Establece los derechos, requisitos y procedimientos para obtener una patente y reforzar la seguridad jurídica.Esta ultima versión simplifica y agiliza el procedimiento.
    \item Real Decreto Legislativo 1/1996, de 12 de abril,\cite{boe--1996-8930} sobre la Ley de la Propiedad Intelectual: Establece los derechos de un obra al autor solo por el hecho de crearlo. 
    \item Real Decreto 192/2023, de 21 de marzo \cite{ministerio_de_sanidad_real_2023}, por el que se regulan los productos sanitarios. 
    \item UNE-EN 60601-1:2008 \cite{UNE2008}, norma que regula los requisitos generales para la seguridad básica y funcionamiento esencial de equipos electromédicos.
    \item UNE-EN ISO 14971:2020 \cite{UNE2020}, norma para la gestión de riesgos de dispositivos médicos.

\end{itemize}

\subsection{2º Fase}
Durante la segunda fase de comercialización y postventa, se deberán incluir las siguientes normas legislativas: 
\begin{itemize}
    \item Ley 3/1991, de 10 de enero \cite{boe--1991-628}, de Competencia Desleal, sobre la promoción y comercialización del producto,realizando una publicidad lícita, veraz y no engañosa.
    \item Ley Orgánica 3/2018, de 5 de diciembre \cite{boe--2018-16673}, de Protección de Datos Personales y garantía de los derechos digitales.
    \item Reglamento (UE) 2016/679 del Parlamento Europeo y del Consejo, de 27 de abril de 2016 \cite{boees}, relativo a la protección de las personas físicas en lo que respecta al tratamiento de datos personales y a la libre circulación de estos datos.
    \item Ley 29/2006, de 26 de julio \cite{boe--2006-13554}, de garantías y uso racional de los medicamentos y productos sanitarios.
\end{itemize}
