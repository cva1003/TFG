\apendice{Documentación de usuario}
En esta sección del anexo se han definido aquellos requisitos necesarios para la ejecución y puesta en marcha del proyecto. 
\section{Requisitos software y hardware para ejecutar el proyecto.}
\subsection{Requisitos del software}
Para el correcto funcionamiento es necesario tener instalado en el ordenador todas las herramientas y programas que se han utilizado. 
\begin{itemize}
    \item Python: Este programa es esencial para que el código se ejecute y se abra el desplegable.
    \item Arduino IDE: Este programa es necesario para tener una comunicación y conexión correcta entre la placa de arduino y el ordenador 
    \item Editor de texto/codigo: No es estrictamente necesario la instalación de un editor de texto/codigo como Visual Estudio Code, pero si puede facilitar el uso tanto al usuario como a posibles profesionales ingenieros para la solución de errores.
    \item Instalación de bibliotecas:
    
    - Pandas: manipulación y analisis de datos estructurados(tablas)
    
    -Pyserial: comunicación con puerto serie
    
    -Kivy: paquete que permite la creación de la interfaz gráfica
    
    -kivymd: paquete que permite la creación de la interfaz gráfica

    -Openpyxl: abrir y modificar archivos excel.
    
    

\begin{table}[]
\centering
\begin{tabular}{|l|l|}
\hline
\rowcolor[HTML]{BFBFBF} 
\textbf{Software} & \textbf{Descripción} \\ \hline
Python & Versión 3.13\\ \hline
Arduino IDE & Versión 2.3.6\\ \hline
Bibliotecas & Versiones más actualizadas\\ \hline
\end{tabular}
\caption{Requisitos Software}
\label{tab:Requisitos_Software}
\end{table}
\end{itemize}
\subsection{Requisitos del hardware}
Para el correcto funcionamiento es necesario el uso del dispositivo creado, un ordenador y el cable USB que conecte el ordenador con la placa de Arduino.

\section{Instalación / Puesta en marcha}
\subsection{Python}

\subsection{Arduino IDE}

\section{Manuales y/o Demostraciones prácticas}




    
     